\chapter{Using PEA in end user mode}
\label{ch:pea_user_mode}
When set to end user mode, the PEA component of Ginan will process each station separately. This mode will allow the estimation of parameters available to users with single receivers. 
\begin{itemize}
	\item Receiver position
	\item Receiver clock offset
	\item Tropospheric delay at receiver location
	\item Ionospheric delay at the receiver location (not yet available)
	\item Carrier phase ambiguities
\end{itemize}
In order to use the PEA in end-user mode, the \textit{ processing\_options : process\_modes : user} parameter needs to be set to \textit{true}.
The results of PEA run in end user mode are printed in the trace files.
Trace file outputs can be activated by setting the \textit{ output\_files : output\_trace} parameter to \textit{true}.
The most commonly used outputs from the PEA used in end-user mode are expected to be: the receiver position, receiver velocity, receiver clocks and tropospheric delays.

\textbf{Receiver position}  results are preceded by the "\$POS" label and thus, in Linux, can be extracted using the command:
 \begin{lstlisting}
$ grep "$POS" <path_to_trace_file>
\end{lstlisting}
the output line for the for receiver position will have 10 comma separated fields with the following format:
\begin{lstlisting}
$POS,2166,278015.000,6,-4052053.0060,4212836.8682,-2545105.0796,0.0245227,0.0231919,0.0163678
\end{lstlisting}
the fields represent, from left to right:
\begin{enumerate}
	\item  "\$POS" label
	\item  GPS week
	\item  GPS TOW in seconds
	\item  Solution type (6 for float PPP, 1 for ambiguity fixed PPP)
	\item  Receiver ECEF X position in meters
	\item  Receiver ECEF Y position in meters
	\item  Receiver ECEF Z position in meters
	\item  Standard deviation of ECEF X positions in meters
	\item  Standard deviation of ECEF X positions in meters
	\item  Standard deviation of ECEF X positions in meters
\end{enumerate}

\textbf{Receiver clock} offset results are preceded by the "\$CLK" label and thus, in Linux, can be extracted using the command:
 \begin{lstlisting}
$ grep "$CLK" <path_to_trace_file>
\end{lstlisting}
the output line for the for receiver position will have 13 comma separated fields with the following format:
\begin{lstlisting}
$CLK,2166,278015.000,6,14,3.1902,0.0000,1.1924,0.0000,0.0860,0.0000,0.0953,0.0000
\end{lstlisting}
the fields represent, from left to right:
\begin{enumerate}
	\item  "\$CLK" label
	\item  GPS week
	\item  GPS TOW in seconds
	\item  Solution type (6 for float PPP, 1 for ambiguity fixed PPP)
	\item  Number of satellites used in the solution
	\item  Receiver clock offset for with respect to GPS clock, in nanoseconds
	\item  Receiver clock offset for with respect to GLONASS clock, in nanoseconds
	\item  Receiver clock offset for with respect to Galileo clock, in nanoseconds
	\item  Receiver clock offset for with respect to Beidou clock, in nanoseconds	
	\item  Standard deviation of clock offset wrt. GPS, in nanoseconds
	\item  Standard deviation of clock offset wrt. GLONASS, in nanoseconds
	\item  Standard deviation of clock offset wrt. Galileo, in nanoseconds
	\item  Standard deviation of clock offset wrt. Beidou, in nanoseconds	
\end{enumerate}
If clock offsets for a particular constellation are not available both the offset and its variance will be set to 0.

\textbf{Tropospheric delays} at the receiver position are preceded by the "\$TROP" label and thus, in Linux, can be extracted using the command:
 \begin{lstlisting}
$ grep "$TROP" <path_to_trace_file>
\end{lstlisting}
the tropospheric delay solutions will be represented to either a single line, with the "\$TROP" or three lines, as follows:
\begin{lstlisting}
$TROP,2166,278015.000,6,14,2.294950,0.0030977
$TROP_N,2166,278015.000,6,14,-0.174797,0.0181385
$TROP_E,2166,278015.000,6,14,-0.223868,0.0250276
\end{lstlisting}
each of the troposphere output line will contain 7 comma separated field, of which the first five are:
\begin{enumerate}
	\item  Label, "\$TROP", "\$TROP\_N" or "\$TROP\_E"
	\item  GPS week
	\item  GPS TOW in seconds
	\item  Solution type (6 for float PPP, 1 for ambiguity fixed PPP)
	\item  Number of satellites used in the solution
\end{enumerate}
The line starting with "\$TROP" contain the Zenith Tropospheric Delay (ZTD) and its standards deviation, both in meters, as their last two fields.  The line starting with "\$TROP\_N" contains the tropospheric delay gradient in north-south direction, and  the line starting with "\$TROP\_E" contains the tropospheric delay gradient in east-west direction.

Configuration files for specific examples have been added to the \textit{examples} folder in the repository. Examples corresponding to end user processing are explained bellow. In order to 

\subsection{Dual frequency PPP with floating ambiguities}
As the end-user processing mode cannot calculate satellite states, the satellite position and clock offset needs to be provided externally.
The PEA supports SP3 formatted satellite position inputs, specified in \textit{input\_files : sp3files}, and RINEX clock files, \textit{input\_files : clkiles}, as satellite clock inputs. 
ANTEX files, with antenna information for both stations and satellites should be provided in  \textit{input\_files : atxfiles} 
SINEX files, with station antennas should be provided in  \textit{input\_files : snxfiles}
RINEX 3.XX navigation files, with broadcast clocks should be provided in  \textit{input\_files : navfiles}
BLQ formatted ocean tide  loading parameters for each station should be provided in \textit{input\_files : blqfiles} if available to correct for OTL, otherwise \textit{processing\_options : tide\_otl} should be set to \textit{false}.
RINEX 3.XX observation files for the network stations should be provided in \textit{station\_data : rnxfiles} (* can be used as a wildcard).\\

The configuration files named \textit{examples/ex11\_pea\_pp\_user\_gps.yaml} and \textit{examples/ex12\_pea\_pp\_user\_gnss.yaml} set the PEA to calculate a post-process end user solution for a static receiver. 
The constellations to be used in processing can be specified in the \textit{processing\_options : process\_sys} field.
The tracking of a moving receiver can be done by setting the \textit{default\_filter\_parameters : stations : pos : proc\_noise} parameter to the maximum expected velocity.
Receiver velocity can also be estimated by setting  \textit{default\_filter\_parameters : stations : pos\_rate : estimate} to \textit{true}.\\

Tropospheric delays are estimated as a combination of hydrostatic and wet components, each component is in turn estimated as the products of the zenith delay and a mapping function.
 If \textit{default\_filter\_parameters : stations : trop : estimate} is set to \textit{true}, the PEA estimates the zenith wet delay.
 If \textit{default\_filter\_parameters : stations : trop\_grad : estimate} is set to \textit{true}, the PEA also estimates azimutal components of tropospheric mapping functions.
The hydrostatic zenith delays and elevation dependent component of mapping functions are calculated based on pre-defined models.
Available models, whic can be selected using the \textit{processing\_options : troposphere : model} parameter, are the GPT2 and VMF3 models.
If using the GPT2 model the path to the necessary grid file needs to be specified in  \textit{processing\_options : troposphere : gpt2grid}
If using the VMF3 model, the tropospheric parameters corresonding to the observatin times need to be provided in a directory specified by \textit{processing\_options : troposphere : vmf3dir}, and the orography file for atmospheric circulation models need to be specified in \textit{processing\_options : troposphere : orography}.\\
 
\subsection{Single frequency PPP}
It is possible to perform end user PPP processing using single frequency data (although at reduced accuracy) by providing extenal Ionospheric delay data.
The configuration files named \textit{examples/ex13\_pea\_pp\_user\_gps\_sf.yaml} set an example to process single frequency observations.
The PEA currently uses IONEX formatted VTEC maps as Ionosphere delay data. 
The path to the IONEX file needs to be specified  in  \textit{input\_files : ionfiles}.
In order for the PEA to use the VTEC maps, the \textit{processing\_options : ionosphere : corr\_mode} parameter should to be set to \textit{total\_electron\_content}.
If provided separatelly, files containing the satellite DCB (either RINEX DCB or bias SINEX) should be specified in \textit{input\_files :dcbfiles} \\

\subsection{Dual frequency PPP with ambiguity resolution}
The PEA (in both network and user processing modes) can be specified to perform ambiguity resolution in an attempt to improve accuracy and convergence times.
In aside from the requirements for floating PPP ambiguities, information on satellite hardware biases needs to be provided order to allow correct ambiguity resolution in end user PEA processing.
For post-process, the PEA use bias SINEX formatted files as input channels for satellite hardware biases.
The bias SINEX file can be specified in \textit{input\_files :bsxfiles}.

The ambiguity resolution process is controlled by the \textit{ambiguity\_resolution\_options} field.
Currently ambiguity resolution is only supported for GPS and Galileo satellited. 
Ambiguity resolution for GPS satellites can be activated by setting the \textit{GPS\_amb\_resol} parameter to \textit{true}.
Ambiguity resolution for Galileo satellites can be activated by setting the \textit{GAL\_amb\_resol} parameter to \textit{true}.
In addition the ambiguity resolution algorithm needs to be specified for both the wide-lane ambiguity (\textit{WL\_mode}) and narrow-lane ambiguities (\textit{NL\_mode}.
For best results, \textit{round} or \textit{iter\_rnd} are recommended for Wide-lane ambiguities and  \textit{lambda\_alt} or \textit{lambda\_bie} is recommended for narrow-lane ambiguities.\\

\subsection{Real-time PPP}
The PEA can also be used to process GNSS data in real-time. 
Real-time processing will make use of RTCM formatted streams for receiver observables and satellite data.
Currently the PEA can only get real-time data by connecting to an NTRIP caster.
An example of such a caster, can be accessed by registering at \href{https://www.auscors.ga.gov.au/}.
The host name, user name and password corresponding to the NTRIP can shold be specified under \textit{station\_data : stream\_root} using the format \textit{http(s)://user:password@hostname/}.
The mountpoint corresponding to station observables need to be listed under \textit{station\_data : obs\_streams}.
Ephemeris streams (broadcast ephemeris and SSR corrections) shold be listed under \textit{station\_data : nav\_streams}.
The PEA support MSM4, MSM5, MSM6 and MSM7 messages for observations, and orbit and clock messages, code bias messages and phase bias messages for GPS and Galileo.\\

Real-time outputs are not yet defined for PEA, the processed receiver solution are printed in real time on the TRACE files.\\